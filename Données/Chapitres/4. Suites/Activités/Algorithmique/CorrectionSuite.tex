% In your .tex file
% !TEX program = pdflatex

\documentclass[a4paper,11pt]{article}
\usepackage{pdflscape}
\usepackage[utf8]{inputenc}
\usepackage[T1]{fontenc}
%\usepackage{fourier} % math & rm
%\usepackage{amsthm,amsfonts,amsmath,amssymb,textcomp}
\usepackage{pst-all,pstricks-add,pst-eucl}
\everymath{\displaystyle}
\usepackage{fp,ifthen}
%\usepackage{color}
%\usepackage{graphicx}
\usepackage{setspace}
\usepackage{array}
\usepackage{tabularx}
\usepackage{supertabular}
\usepackage{hhline}
\usepackage{variations}
\usepackage{enumerate}
\usepackage{pifont}
\usepackage{framed}
\usepackage[fleqn]{amsmath}
\usepackage{amssymb}
\usepackage[framed]{ntheorem}
\usepackage{multicol}
\usepackage{kpfonts}
\usepackage{manfnt}
\usepackage{hyperref}



%\usepackage[hmargin=2.5cm, vmargin=2.5cm]{geometry}
\usepackage{vmargin}          % Pour fixer les marges du document
\setmarginsrb
{1.5cm} 	%marge gauche
{0.5cm} 	  %marge en haut
{1.5cm}     %marge droite
{0.5cm}   %marge en bas
{1cm} 	%hauteur de l'entête
{0.5cm}   %distance entre l'entête et le texte
{1cm} 	  %hauteur du pied de page
{0.5cm}     %distance entre le texte et le pied de page

\newcommand{\R}{\mathbb{R}}
\newcommand{\N}{\mathbb{N}}
%\newcommand{\D}{\mathbb{D}}
\newcommand{\Z}{\mathbb{Z}}
\newcommand{\Q}{\mathbb{Q}}
\newcommand{\C}{\mathbb{C}}
\newcommand{\e}{\text{e}}
\newcommand{\dx}{\text{d}x}
\newcommand{\vect}[1]{\mathchoice%
  {\overrightarrow{\displaystyle\mathstrut#1\,\,}}%
  {\overrightarrow{\textstyle\mathstrut#1\,\,}}%
  {\overrightarrow{\scriptstyle\mathstrut#1\,\,}}%
  {\overrightarrow{\scriptscriptstyle\mathstrut#1\,\,}}}
\newcommand\arraybslash{\let\\\@arraycr}
\renewcommand{\theenumi}{\textbf{\arabic{enumi}}}
\renewcommand{\labelenumi}{\textbf{\theenumi.}}
\renewcommand{\theenumii}{\textbf{\alph{enumii}}}
\renewcommand{\labelenumii}{\textbf{\theenumii.}}
\renewcommand{\and}{\wedge}

\theoremstyle{break}
\theorembodyfont{\upshape}
\newframedtheorem{Theo}{Théorème}
\newframedtheorem{Prop}{Propriété}
\newframedtheorem{Def}{Définition}

\newtheorem{Term}{Terminologie}
\newtheorem{Rq}{Remarque}
\newtheorem{Ex}{Exemple}
\newtheorem{exo}{Exercice}
\renewcommand{\theexo}{\empty{}} 

%\theorembodyfont{\small \sffamily}
%\newtheorem{sol}{solution}

\newenvironment{sol}% 
{\def\FrameCommand{\hspace{0.5cm} {\color{black} \vrule width 1pt} \hspace{-0.7cm}}%
  \framed {\advance\hsize-\width}
  \noindent \small \sffamily  %\underline{Solution :}%\\
}%
{\endframed}

\newrgbcolor{vert}{0 0.4 0}
\newrgbcolor{bistre}{1 .50 .30}
\setlength\tabcolsep{1mm}
\renewcommand\arraystretch{1.3}

\everymath{\displaystyle}
\hyphenpenalty 10000 %supprime toutes les césures
%\setcounter{secnumdepth}{0}
%\newcounter{saveenum}

\usepackage[frenchb]{babel}
\usepackage{fancyhdr,lastpage}
\usepackage{fancybox}

%\headheight 15.0 pt
\fancyhead[L]{Correction Algorithmique sur les suites}
\fancyhead[C]{}
\fancyhead[R]{Année 2015-2016}
\fancyfoot[L]{{\scriptsize\textsl{Cité scolaire de Lorgues}}}
%\fancyfoot[C]{\scriptsize\thepage}
%\fancyfoot[C]{\scriptsize\thepage/\pageref{LastPage}}

\title{Algorithmique : Calcul de termes et de sommes de termes d'une suite.}
\author{}
\date{}

%\pagestyle{empty}
\pagestyle{fancy}
%\usepackage[np]{numprint}

\renewcommand\arraystretch{1.8}

\newcounter{numero}
%\newcommand{\exo}{
%  \addtocounter{numero}{1}%
%  \textbf{\underline{Exercice \arabic{numero}:}}\quad}

\frenchbsetup{StandardEnumerateEnv=true}
\usepackage{etex}
\usepackage{tikz,tkz-tab}


\newframedtheorem{Dev}{Devoirs}
\renewcommand{\theDev}{\empty{}} 

\newcommand{\dm}{
  \textbf{\underline{Devoir à la maison:}}\quad \vspace{0.5cm}}
  
  

\begin{document}

\maketitle
  \setlength{\unitlength}{1mm}
  \setlength\parindent{0mm}
  
  
  %\exo
  ~
  \medskip
  
  \iffalse

 \fi
    
  
  \begin{itemize}
   \item  Soit $(u_n)$ la suite définie par $u_n=n^3$ pour tout entier $n$.
   \item Soit $(v_n)$ la suite définie par récurrence par $v_0=4$ et pour tout entier $n$, $v_{n+1}=2v_n+1$.
  \end{itemize}
  
  \begin{enumerate}
  \item
  Réaliser un programme permettant de calculer les 30 premiers termes de $(u_n)$. 
  
    \item
  Réaliser un programme permettant de calculer $v_{30}$. 
  
  \item 
  Réaliser un programme permettant de trouver le plus petit entier $n$ pour lequel $u_n>10^{10}$.
  
   \item 
  Réaliser un programme permettant de trouver le plus petit entier $n$ pour lequel $v_n>10^{10}$.
  
  \item
  Réaliser un programme permettant de calculer la somme des 30 premiers termes de $(u_n)$. 
  
  \item 
  Réaliser un programme permettant de calculer les 30 premiers termes de $(v_n)$. 
  
  \end{enumerate}
  
  \newpage
  
  \begin{itemize}
   \item  Soit $(u_n)$ la suite définie par $u_n=n^3$ pour tout entier $n$.
   \item Soit $(v_n)$ la suite définie par récurrence par $v_0=4$ et pour tout entier $n$, $v_{n+1}=2v_n+1$.
  \end{itemize}
  
  \begin{enumerate}
  \item
  Réaliser un programme permettant de calculer les 30 premiers termes de $(u_n)$.
  
  \vspace{0.5cm}
  
  Pour $N$ allant de 0 à 29
  
  $R$ prend la valeur $N^3$
  
  Afficher $R$
  
  FinPour
  \vspace{0.5cm}
  
    \item
  Réaliser un programme permettant de calculer $v_{30}$. 
  
  \vspace{0.5cm}
  
  $V$ prend la valeur 4 
  
  Pour $N$ allant de 1 à 30
  
  $V$ prend la valeur $2V+1$
  
  FinPour
  
  Afficher $V$
  
  \vspace{0.5cm}
  
  \item 
  Réaliser un programme permettant de trouver le plus petit entier $n$ pour lequel $u_n>10^{10}$.
  
  
  \vspace{0.5cm}
  
  $N$ prend la valeur 0
  
  Tant que $N^3 \leq 10^{10}$
  
  $N$ prend la valeur $N+1$
  
  FinTantque
  
  Afficher $N$
  
  \vspace{0.5cm}
  
   \item 
  Réaliser un programme permettant de trouver le plus petit entier $n$ pour lequel $v_n>10^{10}$.
  
  \vspace{0.5cm}
  
  $N$ prend la valeur 0
  
  $V$ prend la valeur 4
  
  Tant que $V \leq 10^{10}$
  
  $N$ prend la valeur $N+1$
  
  $V$ prend la valeur $2V+1$
  
  FinTantque
  
  Afficher $N$
  
  \vspace{0.5cm}
  
  \item
  Réaliser un programme permettant de calculer la somme des 30 premiers termes de $(u_n)$. 
  
  
  \vspace{0.5cm}
  
  $S$ prend la valeur 0
  
  Pour N allant de 1 à 29
  
  $S$ prend la valeur $S+N^3$
    
  FinPour
  
  Afficher $S$
  
  \vspace{0.5cm}
   
  \item 
  Réaliser un programme permettant de calculer les 30 premiers termes de $(v_n)$. 
  
  
  \vspace{0.5cm}
  
  V prend la valeur 4 
  
  Pour N allant de 1 à 30
  
  V prend la valeur 2V+1
  
  Afficher V
  
  FinPour
  
  \vspace{1cm}
  
  \end{enumerate} 
 
  
\end{document}

 


