\documentclass[a4paper,11pt]{article}
\usepackage{pdflscape}
\usepackage[utf8]{inputenc}
\usepackage[T1]{fontenc}
%\usepackage{fourier} % math & rm
%\usepackage{amsthm,amsfonts,amsmath,amssymb,textcomp}
\usepackage{pst-all,pstricks-add,pst-eucl}
\everymath{\displaystyle}
\usepackage{fp,ifthen}
%\usepackage{color}
%\usepackage{graphicx}
\usepackage{setspace}
\usepackage{array}
\usepackage{tabularx}
\usepackage{supertabular}
\usepackage{hhline}
\usepackage{variations}
\usepackage{enumerate}
\usepackage{pifont}
\usepackage{framed}
\usepackage[fleqn]{amsmath}
\usepackage{amssymb}
\usepackage[framed]{ntheorem}
\usepackage{multicol}
\usepackage{kpfonts}
\usepackage{manfnt}

%\usepackage[hmargin=2.5cm, vmargin=2.5cm]{geometry}
\usepackage{vmargin}          % Pour fixer les marges du document
\setmarginsrb
{1.5cm} 	%marge gauche
{0.5cm} 	  %marge en haut
{1.5cm}     %marge droite
{0.5cm}   %marge en bas
{1cm} 	%hauteur de l'entête
{0.5cm}   %distance entre l'entête et le texte
{1cm} 	  %hauteur du pied de page
{0.5cm}     %distance entre le texte et le pied de page

\newcommand{\R}{\mathbb{R}}
\newcommand{\N}{\mathbb{N}}
%\newcommand{\D}{\mathbb{D}}
\newcommand{\Z}{\mathbb{Z}}
\newcommand{\Q}{\mathbb{Q}}
\newcommand{\C}{\mathbb{C}}
\newcommand{\e}{\text{e}}
\newcommand{\dx}{\text{d}x}
\newcommand{\vect}[1]{\mathchoice%
  {\overrightarrow{\displaystyle\mathstrut#1\,\,}}%
  {\overrightarrow{\textstyle\mathstrut#1\,\,}}%
  {\overrightarrow{\scriptstyle\mathstrut#1\,\,}}%
  {\overrightarrow{\scriptscriptstyle\mathstrut#1\,\,}}}
\newcommand\arraybslash{\let\\\@arraycr}
\renewcommand{\theenumi}{\textbf{\arabic{enumi}}}
\renewcommand{\labelenumi}{\textbf{\theenumi.}}
\renewcommand{\theenumii}{\textbf{\alph{enumii}}}
\renewcommand{\labelenumii}{\textbf{\theenumii.}}
\renewcommand{\and}{\wedge}

\theoremstyle{break}
\theorembodyfont{\upshape}
\newframedtheorem{Theo}{Théorème}
\newframedtheorem{Prop}{Propriété}
\newframedtheorem{Def}{Définition}

\newtheorem{Term}{Terminologie}
\newtheorem{Rq}{Remarque}
\newtheorem{Ex}{Exemple}
%\newtheorem{exo}{Exercice}

%\theorembodyfont{\small \sffamily}
%\newtheorem{sol}{solution}

\newenvironment{sol}% 
{\def\FrameCommand{\hspace{0.5cm} {\color{black} \vrule width 1pt} \hspace{-0.7cm}}%
  \framed {\advance\hsize-\width}
  \noindent \small \sffamily  %\underline{Solution :}%\\
}%
{\endframed}

\newrgbcolor{vert}{0 0.4 0}
\newrgbcolor{bistre}{1 .50 .30}
\setlength\tabcolsep{1mm}
\renewcommand\arraystretch{1.3}

\everymath{\displaystyle}
\hyphenpenalty 10000 %supprime toutes les césures
%\setcounter{secnumdepth}{0}
%\newcounter{saveenum}

\usepackage[frenchb]{babel}
\usepackage{fancyhdr,lastpage}
\usepackage{fancybox}

%\headheight 15.0 pt
\fancyhead[L]{Avec des suites et des algorithmes}
\fancyhead[C]{Contrôle sujet A}
\fancyhead[R]{Jeudi 26 janvier 2017}
%\fancyfoot[L]{{\scriptsize\textsl{Cité scolaire de Lorgues}}}
\fancyfoot[C]{}
%\fancyfoot[C]{\scriptsize\thepage}
%\fancyfoot[C]{\scriptsize\thepage/\pageref{LastPage}}

\title{}
\author{}
\date{}

%\pagestyle{empty}
\pagestyle{fancy}
\usepackage[np]{numprint}

\renewcommand\arraystretch{1.8}

\newcounter{numero}
\newcommand{\exo}{
  \addtocounter{numero}{1}%
  \textbf{\underline{Exercice \arabic{numero}:}}\quad}

\frenchbsetup{StandardEnumerateEnv=true}
\usepackage{etex}
\usepackage{pgf,tikz,tkz-tab}
\usepackage{comment}
\includecomment{correction}
%\renewcommand{correction}{correction}
%\excludecomment{correction}
\usepackage{array}


\begin{document}
  \setlength{\unitlength}{1mm}
  \setlength\parindent{0mm}
  
  %\thispagestyle{empty}
  %\exo
  
  
  \vspace{1cm}
  ~
  
  \begin{exo}(5 points)
~
      \vspace{0.25cm}
      
      Réaliser un algorithme permettant de calculer et afficher le plus petit entier naturel n
      
      pour lequel $2^n>1073741824$.
      
\begin{correction}

Correction

      n prend la valeur 0
 
      tant que $2^n \leq 1073741824$ faire
      
      n prend la valeur n+1
      
      fin tant que
      
      Afficher n
      
\end{correction}      
      
      
\end{exo}  

     ~
  \vspace{0.5cm}

  
  \begin{exo}(5 points)
  ~
      \vspace{0.25cm}
      
 %Variation d'une suite définie explicitement par un trinôme du second degré.
 
 \begin{enumerate}
 
 \item  Calculer la dérivée $f'(x)$ de la fonction
 $f(x)=\frac{4}{3}x^3+10 x^2 + 24 x +100$  
 \item Décomposer $f'(x)$ en un produit de facteurs de degré $1$ et étudier son signe.  
 \item Montrer que la suite $(u_n)$ définie pour tout entier naturel par 
 $u_n=\frac{4}{3}n^3+10 n^2 + 24 n +100$ est strictement croissante. 
   
 \end{enumerate}

\begin{correction}

Correction
\begin{enumerate}
 \item $f'(x)=4 x^2+20 x+24$ (1 pt)
 \item $f'(x)=4(x+3)(x+2)$, en effet, $\Delta=16, x_1= \frac{-20-4}{8}=-3, x_2=\frac{-20+4}{8}=-2$. (2 pts)
 
 $f'(x)>0$ sur $]-\infty;-3[ \cup ]-2;+\infty[$, $f'(x)<0$ sur $]-3;-2[$.(1 pt)
 \item La fonction $f$ est strictement croissante sur l'intervalle $[0;+\infty[$ donc la suite $(u_n=f(n))$ 
 est strictement croissante. (1 pt)
\end{enumerate}


  
  
\end{correction}

  \end{exo}
  
   ~
  \vspace{0.5cm}
  
   \begin{exo}(7 points)
   ~
      \vspace{0.25cm}
   
  %Calculs de termes d'une suite géométrique et forme explicite d'une suite aithmétique
  
  \begin{enumerate}
   \item Soit $(u_n)$ une suite arithmétique telle que $u_3=2$ et $u_6=13$.
   
   Calculer la forme explicite de la suite $(u_n)$ ainsi que $u_{15}$. 
   
   \item Calculer le 5ème terme de la suite géométrique de premier terme $v_0=3$ et de raison 
   $\frac{1}{\sqrt 3}$.
   
   \item Démontrer que la suite$(w_n)$ définie pour tout entier $n$ par $w_n=11(9)^n$ est géométrique
   et donner sa raison et son premier terme.
   
   \item Démontrer que la suite $(x_n)$ défine pour tout entier $n>0$ par $x_n=\frac{1}{n^2}$ n'est ni
   arithmétique ni géométrique. 
  \end{enumerate}

\begin{correction}

Correction
\begin{enumerate}
 \item $u_6-u_3=11=3r$ d'o\`u $r=\frac{11}{3}$, $u_n=u_3+(n-3)r=2+(n-3)\frac{11}{3}$ (1 pt)
 
 $u_{15}=2+(15-3)\frac{11}{3}=2+\frac{12 \times 11}{3}=46$ (1 pt)
 \item $v_{4}=v_0(\frac{1}{\sqrt{3}})^{(4-0)}=3(\frac{1}{3})^2=\frac{1}{3}$. (1 pt)
 \item $w_{n+1}=11(9)^{n+1}=11(9)^n\times 9=w_n \times 9$. (1 pt)
 La suite est bien g\'eom\'etrique de raison $9$ 
 et de premier terme $w_0=11$. (1 pt)
 \item $x_1=1$, $x_2=\frac{1}{4}=x_1-\frac{3}{4}=x_1 \times \frac{1}{4}$. 
 $x_3=\frac{1}{9} \neq -\frac{1}{2}=x_2-\frac{3}{4}$ et $x_3 \neq \frac{1}{16} = x_2 \times \frac{1}{4}$.
 Ainsi, $(u_n)$ n'est ni arithm\'etique ni 
 g\'eom\'etrique. (2 pts)
\end{enumerate}

\end{correction}
  \end{exo}
    
     ~
  \vspace{0.5cm}
 

\begin{exo}(3 points)

Soit $(u_n)$ la suite définie par récurrence par $u_0=3$ et pour tout entier $n$ par 
$u_{n+1}=2 u_n +5$. On admet que pour tout entier naturel $n$, $u_n>0$.

\begin{enumerate}
  \item Démontrer que le suite $(u_n)$ est strictement croissante.
  \item Trouver une forme explicite pour la suite $(u_n)$. On pourra introduire la suite auxiliaire
  $v_n=u_n+5$.
\end{enumerate}

\begin{correction}
\begin{enumerate}
 \item $u_{n+1}-u_n=u_n+5>0$ pour tout entier naturel. $(u_n)$ est donc strictement croissante. (1 pt)
 \item $v_{n+1}=u_{n+1}+5=2 u_n+5+5=2 u_n +10=2(u_n+5)=2 v_n$. La suite $(v_n)$ est g\'eom\'etrique
  de raison 2 et de premier terme $v_0=u_0+5=8$. D'o\`u $v_n=8 \times 2^n$ et $u_n=8 \times 2^n-5$. (2 pts)
\end{enumerate}


\end{correction}

\begin{correction}

\end{correction}


\end{exo}

\end{document}

\begin{exo}(5 points)
~
      \vspace{0.25cm}
      
Lors d'une saison footballistique comprenant 38 matchs, deux équipes
ont encaissé des buts selon les répartitions suivantes:

\begin{center}

\begin{tabular}{l l}
 
 Equipe A: 

\begin{tabular}{|l|c|c|c|c|c|}
  \hline
  Nombres de buts  & 0  & 1 &3 & 4 & 5 \\
  \hline
  Nombre de matchs & 23 & 7 &5 & 1 &2 \\
  
  \hline
\end{tabular}

 &

 Equipe B:

\begin{tabular}{|l|c|c|c|c|}
  \hline
  Nombres de buts  & 0  & 1 &3 & 6 \\
  \hline
  Nombre de matchs & 19 & 12 &6 & 1  \\
  
  \hline
\end{tabular}

 \end{tabular}

\end{center} 
\vspace{0.5cm}
\begin{enumerate}
 \item Après avoir écrit la formule avec un symbole de somme, calculer le nombre moyen de buts
  encaissés par match pour ces deux équipes.
  \item Après avoir écrit la formule avec un symbole de somme, calculer l'écart-type de la série
  des nombres de buts encaissés par match pour chacune des deux équipes au dixième de but près.
  \item Au vu de ces résultats, quelle équipe possède des performances défensives les plus 
  irrégulières ?
\end{enumerate}



\begin{correction}

\begin{enumerate}
 \item Soit $(a,n)$ (resp $(b,m)$) la s\'erie statistique associ\'ee \'a l'\'equipe $1$ (resp. 2).
 
 $\overline{a}=\sum_{i=1}^5 \frac{n_i a_i}{38}=\frac{36}{38}=\frac{18}{19}$. (1 pt)
 et $\overline{b}=\sum_{i=1}^4 \frac{m_i b_i}{38}=\frac{36}{38}=\frac{18}{19}$. (1 pt)
 \item $\sigma_{a}=\sqrt{\sum_{i=1}^5 \frac{n_i (a_i-\overline{a})^2}{38}} \simeq 1.5$ (1 pt)
 et $\sigma_{b}=\sqrt{\sum_{i=1}^4 \frac{m_i (b_i-\overline{b})^2}{38}}\simeq 1.3$. (1 pt)
  \item L'\'equipe B poss\`ede les performances d\'efensives les plus r\' eguli\`eres sur cette saison.
\end{enumerate}

\end{correction}

\end{exo}
