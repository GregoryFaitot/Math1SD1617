\documentclass[a4paper,11pt]{article}
\usepackage{pdflscape}
\usepackage[utf8]{inputenc}
\usepackage[T1]{fontenc}
%\usepackage{fourier} % math & rm
%\usepackage{amsthm,amsfonts,amsmath,amssymb,textcomp}
\usepackage{pst-all,pstricks-add,pst-eucl}
\everymath{\displaystyle}
\usepackage{fp,ifthen}
%\usepackage{color}
%\usepackage{graphicx}
\usepackage{setspace}
\usepackage{array}
\usepackage{tabularx}
\usepackage{supertabular}
\usepackage{hhline}
\usepackage{variations}
\usepackage{enumerate}
\usepackage{pifont}
\usepackage{framed}
\usepackage[fleqn]{amsmath}
\usepackage{amssymb}
\usepackage[framed]{ntheorem}
\usepackage{multicol}
\usepackage{kpfonts}
\usepackage{manfnt}

%\usepackage[hmargin=2.5cm, vmargin=2.5cm]{geometry}
\usepackage{vmargin}          % Pour fixer les marges du document
\setmarginsrb
{1.5cm} 	%marge gauche
{0.5cm} 	  %marge en haut
{1.5cm}     %marge droite
{0.5cm}   %marge en bas
{1cm} 	%hauteur de l'entête
{0.5cm}   %distance entre l'entête et le texte
{1cm} 	  %hauteur du pied de page
{0.5cm}     %distance entre le texte et le pied de page

\newcommand{\R}{\mathbb{R}}
\newcommand{\N}{\mathbb{N}}
%\newcommand{\D}{\mathbb{D}}
\newcommand{\Z}{\mathbb{Z}}
\newcommand{\Q}{\mathbb{Q}}
\newcommand{\C}{\mathbb{C}}
\newcommand{\e}{\text{e}}
\newcommand{\dx}{\text{d}x}
\newcommand{\vect}[1]{\mathchoice%
  {\overrightarrow{\displaystyle\mathstrut#1\,\,}}%
  {\overrightarrow{\textstyle\mathstrut#1\,\,}}%
  {\overrightarrow{\scriptstyle\mathstrut#1\,\,}}%
  {\overrightarrow{\scriptscriptstyle\mathstrut#1\,\,}}}
\newcommand\arraybslash{\let\\\@arraycr}
\renewcommand{\theenumi}{\textbf{\arabic{enumi}}}
\renewcommand{\labelenumi}{\textbf{\theenumi.}}
\renewcommand{\theenumii}{\textbf{\alph{enumii}}}
\renewcommand{\labelenumii}{\textbf{\theenumii.}}
\renewcommand{\and}{\wedge}

\theoremstyle{break}
\theorembodyfont{\upshape}
\newframedtheorem{Theo}{Théorème}
\newframedtheorem{Prop}{Propriété}
\newframedtheorem{Def}{Définition}

\newtheorem{Term}{Terminologie}
\newtheorem{Rq}{Remarque}
\newtheorem{Ex}{Exemple}
%\newtheorem{exo}{Exercice}

%\theorembodyfont{\small \sffamily}
%\newtheorem{sol}{solution}

\newenvironment{sol}% 
{\def\FrameCommand{\hspace{0.5cm} {\color{black} \vrule width 1pt} \hspace{-0.7cm}}%
  \framed {\advance\hsize-\width}
  \noindent \small \sffamily  %\underline{Solution :}%\\
}%
{\endframed}

\newrgbcolor{vert}{0 0.4 0}
\newrgbcolor{bistre}{1 .50 .30}
\setlength\tabcolsep{1mm}
\renewcommand\arraystretch{1.3}

\everymath{\displaystyle}
\hyphenpenalty 10000 %supprime toutes les césures
%\setcounter{secnumdepth}{0}
%\newcounter{saveenum}

\usepackage[frenchb]{babel}
\usepackage{fancyhdr,lastpage}
\usepackage{fancybox}

%\headheight 15.0 pt
\fancyhead[L]{Avec des suites}
\fancyhead[C]{Contrôle sujet C}
\fancyhead[R]{Jeudi 2 mars 2017}
%\fancyfoot[L]{{\scriptsize\textsl{Cité scolaire de Lorgues}}}
\fancyfoot[C]{}
%\fancyfoot[C]{\scriptsize\thepage}
%\fancyfoot[C]{\scriptsize\thepage/\pageref{LastPage}}

\title{}
\author{}
\date{}

%\pagestyle{empty}
\pagestyle{fancy}
\usepackage[np]{numprint}

\renewcommand\arraystretch{1.8}

\newcounter{numero}
\newcommand{\exo}{
  \addtocounter{numero}{1}%
  \textbf{\underline{Exercice \arabic{numero}:}}\quad}

\frenchbsetup{StandardEnumerateEnv=true}
\usepackage{etex}
\usepackage{pgf,tikz,tkz-tab}
\usepackage{comment}
\includecomment{correction}
%\renewcommand{correction}{correction}
%\excludecomment{correction}
\usepackage{array}


\begin{document}
  \setlength{\unitlength}{1mm}
  \setlength\parindent{0mm}
  
  %\thispagestyle{empty}
  %\exo
  
  \vspace{1cm}
  ~
  
  \begin{exo}(5 points)
  ~
      \vspace{0.25cm}
      
 %Variation d'une suite définie explicitement par un trinôme du second degré.
 
 \begin{enumerate}
 
 \item  Calculer la dérivée $f'(x)$ de la fonction
 $f(x)=x^3+\frac{9}{2} x^2 + 6 x +4$  
 \item Décomposer $f'(x)$ en un produit de facteurs de degré $1$ et étudier son signe.  
 \item Montrer que la suite $(u_n)$ définie pour tout entier naturel par 
 $u_n=n^3+\frac{9}{2} n^2 + 6 n +4$ est strictement croissante. 
   
 \end{enumerate}

\begin{correction}

Correction
\begin{enumerate}
 \item $f'(x)=3 x^2+9 x+6$ (1 pt)
 \item $f'(x)=3(x+1)(x+2)$, en effet, $\Delta=9, x_1= \frac{-9-3}{6}=-2, x_2=\frac{-9+3}{6}=-1$. (2 pts)
 
 $f'(x)>0$ sur $]-\infty;-2[ \cup ]-1;+\infty[$, $f'(x)<0$ sur $]-2-1[$.(1 pt)
 \item La fonction $f$ est strictement croissante sur l'intervalle $[0;+\infty[$ donc la suite $(u_n=f(n))$ 
 est strictement croissante. (1 pt)
\end{enumerate}


  
  
\end{correction}

  \end{exo}
  
   ~
  \vspace{0.5cm}
  
   \begin{exo}(8 points)
   ~
      \vspace{0.25cm}
   
  %Calculs de termes d'une suite géométrique et forme explicite d'une suite aithmétique
  
  \begin{enumerate}
   \item Soit $(u_n)$ une suite arithmétique de raison 3 avec $u_0=4$.
   
   Calculer la somme $S$ des 26 premiers termes de la suite $(u_n)$. 
   
   \item Démontrer que la suite $(w_n)$ définie pour tout entier $n$ par $w_n=3(5)^n$ est géométrique
   et donner sa raison et son premier terme. 
   
   \item Démontrer que la suite $(x_n)$ défine pour tout entier $n$ par $x_n=n^2$ n'est ni
   arithmétique ni géométrique.
   
   \item Calculer la somme des 10 premiers termes de la suite géométrique de premier terme $v_0=3$ et de raison $2$.
  \end{enumerate}

\begin{correction}

Correction
\begin{enumerate}
 \item $2S=(u_0+u_{25})\times 26=(4+(4+3*25))\times 26=83 \times 26=2158$ d'o\`u $S=1079$. (2 pts)
 \item $w_{n+1}=3(5)^{n+1}=3(5)^n\times 5=w_n \times 5$. (1 pt)
 La suite est bien g\'eom\'etrique de raison $5$ 
 et de premier terme $w_0=3$. (1 pt)
 \item $x_1=1$, $x_2=4=x_1+3=x_1 \times 4$. 
 $x_3=9 \neq 7=x_2+3$ et $x_3 \neq 16 = x_2 \times 4$. Ainsi, $(u_n)$ n'est ni arithm\'etique ni 
 g\'eom\'etrique. (2 pts)
 
 \item $v_{0}+...+v_9=v_0({2}^{0}+...+{2}^{9})=v_0 \frac{1-{2}^{10}}{1-2}=3\frac{{2}^{10}-1}{2-1}=3\times 1023=3069$. (2 pts)
\end{enumerate}

\end{correction}
  \end{exo}
    
     ~
  \vspace{0.5cm}
  
    
  \begin{exo}(4 points)
~
      \vspace{0.25cm}
      
      Réaliser un algorithme permettant de calculer et afficher les inverses des 40 premières entiers (de $\frac{1}{1}$ à $\frac{1}{40}$).
            
      
\begin{correction}

Correction

   Pour n allant de 1 \`a 40 faire
   
      x prend la valeur $\frac{1}{n}$
      
      Afficher x
      
   fin pour
      
      
      
\end{correction}      
      
      
\end{exo}  
    

 ~
  \vspace{1cm}
  

\begin{exo}(3 points)

Soit $(u_n)$ la suite définie par récurrence par $u_0=2$ et pour tout entier $n$ par 
$u_{n+1}=2 u_n +1$. On admet que pour tout entier naturel $n$, $u_n>0$.

\begin{enumerate}
  \item Démontrer que le suite $(u_n)$ est strictement croissante.
  \item Trouver une forme explicite pour la suite $(u_n)$. On pourra introduire la suite auxiliaire
  $v_n=u_n+1$.
\end{enumerate}

\begin{correction}
\begin{enumerate}
 \item $u_{n+1}-u_n=u_n+1>0$ pour tout entier naturel. $(u_n)$ est donc strictement croissante. (1 pt)
 \item $v_{n+1}=u_{n+1}+1=2 u_n+1+1=2 u_n +2=2(u_n+1)=2 v_n$. La suite $(v_n)$ est g\'eom\'etrique
  de raison 2 et de premier terme $v_0=u_0+1=3$. D'o\`u $v_n=3 \times 2^n$ et $u_n=3 \times 2^n-1$. (2 pts)
\end{enumerate}


\end{correction}

\begin{correction}

\end{correction}


\end{exo}


    
\end{document}
