\documentclass[a4paper,11pt]{article}
\usepackage{pdflscape}
\usepackage[utf8]{inputenc}
\usepackage[T1]{fontenc}
%\usepackage{fourier} % math & rm
%\usepackage{amsthm,amsfonts,amsmath,amssymb,textcomp}
\usepackage{pst-all,pstricks-add,pst-eucl}
\everymath{\displaystyle}
\usepackage{fp,ifthen}
%\usepackage{color}
%\usepackage{graphicx}
\usepackage{setspace}
\usepackage{array}
\usepackage{tabularx}
\usepackage{supertabular}
\usepackage{hhline}
\usepackage{variations}
\usepackage{enumerate}
\usepackage{pifont}
\usepackage{framed}
\usepackage[fleqn]{amsmath}
\usepackage{amssymb}
\usepackage[framed]{ntheorem}
\usepackage{multicol}
\usepackage{kpfonts}
\usepackage{manfnt}

%\usepackage[hmargin=2.5cm, vmargin=2.5cm]{geometry}
\usepackage{vmargin}          % Pour fixer les marges du document
\setmarginsrb
{1.5cm} 	%marge gauche
{0.5cm} 	  %marge en haut
{1.5cm}     %marge droite
{0.5cm}   %marge en bas
{1cm} 	%hauteur de l'entête
{0.5cm}   %distance entre l'entête et le texte
{1cm} 	  %hauteur du pied de page
{0.5cm}     %distance entre le texte et le pied de page

\newcommand{\R}{\mathbb{R}}
\newcommand{\N}{\mathbb{N}}
%\newcommand{\D}{\mathbb{D}}
\newcommand{\Z}{\mathbb{Z}}
\newcommand{\Q}{\mathbb{Q}}
\newcommand{\C}{\mathbb{C}}
\newcommand{\e}{\text{e}}
\newcommand{\dx}{\text{d}x}
\newcommand{\vect}[1]{\mathchoice%
  {\overrightarrow{\displaystyle\mathstrut#1\,\,}}%
  {\overrightarrow{\textstyle\mathstrut#1\,\,}}%
  {\overrightarrow{\scriptstyle\mathstrut#1\,\,}}%
  {\overrightarrow{\scriptscriptstyle\mathstrut#1\,\,}}}
\newcommand\arraybslash{\let\\\@arraycr}
\renewcommand{\theenumi}{\textbf{\arabic{enumi}}}
\renewcommand{\labelenumi}{\textbf{\theenumi.}}
\renewcommand{\theenumii}{\textbf{\alph{enumii}}}
\renewcommand{\labelenumii}{\textbf{\theenumii.}}
\renewcommand{\and}{\wedge}

\theoremstyle{break}
\theorembodyfont{\upshape}
\newframedtheorem{Theo}{Théorème}
\newframedtheorem{Prop}{Propriété}
\newframedtheorem{Def}{Définition}

\newtheorem{Term}{Terminologie}
\newtheorem{Rq}{Remarque}
\newtheorem{Ex}{Exemple}
%\newtheorem{exo}{Exercice}

%\theorembodyfont{\small \sffamily}
%\newtheorem{sol}{solution}

\newenvironment{sol}% 
{\def\FrameCommand{\hspace{0.5cm} {\color{black} \vrule width 1pt} \hspace{-0.7cm}}%
  \framed {\advance\hsize-\width}
  \noindent \small \sffamily  %\underline{Solution :}%\\
}%
{\endframed}

\newrgbcolor{vert}{0 0.4 0}
\newrgbcolor{bistre}{1 .50 .30}
\setlength\tabcolsep{1mm}
\renewcommand\arraystretch{1.3}

\everymath{\displaystyle}
\hyphenpenalty 10000 %supprime toutes les césures
%\setcounter{secnumdepth}{0}
%\newcounter{saveenum}

\usepackage[frenchb]{babel}
\usepackage{fancyhdr,lastpage}
\usepackage{fancybox}

%\headheight 15.0 pt
\fancyhead[L]{Second degré}
\fancyhead[C]{Contrôle sujet A}
\fancyhead[R]{Mardi 27 septembre 2016}
%\fancyfoot[L]{{\scriptsize\textsl{Thomas Gire Cité scolaire de Lorgues}}}
\fancyfoot[C]{}
%\fancyfoot[C]{\scriptsize\thepage}
%\fancyfoot[C]{\scriptsize\thepage/\pageref{LastPage}}

\title{}
\author{}
\date{}

%\pagestyle{empty}
\pagestyle{fancy}
\usepackage[np]{numprint}

\renewcommand\arraystretch{1.8}

\newcounter{numero}
\newcommand{\exo}{
  \addtocounter{numero}{1}%
  \textbf{\underline{Exercice \arabic{numero}:}}\quad}

\frenchbsetup{StandardEnumerateEnv=true}
\usepackage{etex}
\usepackage{pgf,tikz,tkz-tab}
\usepackage{comment}
\includecomment{correction}
%\renewcommand{correction}{correction}
%\excludecomment{correction}
\usepackage{array}


\begin{document}
  \setlength{\unitlength}{1mm}
  \setlength\parindent{0mm}
  
  %\thispagestyle{empty}
  %\exo
  
  \vspace{1.5cm}
  ~
  
  \begin{exo}(6 points)
   \begin{enumerate}
    \item Développer et réduire l'expression $(2x+2)(x+1)+(x-3)^2-4x-35$
    
\begin{correction}
    
   $=2x^2+2 x+2 x+2+x^2-6x+9-4x-35=3x^2-6x-24$ (1 pt)

\end{correction}

    
    \item Calculer le discriminant du trinôme $g(x)=3x^2-6x-24$.
    
\begin{correction}
 
 $\Delta=b^2-4ac=(-6)^2-4(3)(-24)=36+36 \times 8=36*9=18^2$ (1 pt)
 
 
\end{correction}
    
    
    \item Combien le trinôme $g(x)$ admet-il de racines ? Calculer toutes ses racines.

\begin{correction}
 
 Comme $\Delta>0$, $g(x)$ admet deux racines. (1 pt)
    
    $x_1=\frac{-b-\sqrt{\Delta}}{2a}=\frac{-(-6)-18}{2 \times 3}=\frac{-12}{6}=-2$ (1 pt)
    et 
    $x_2=\frac{-(-6)+18}{2 \times 3}=4$ (1 pt)
 
\end{correction}

    \item Donner si possible la forme factorisée de $g(x)$.
    
\begin{correction}
 
$g(x)=3(x-4)(x+2)$ (1 pt)


 
\end{correction}


\end{enumerate}
\end{exo}
  
 
  \vspace{0.5cm}
  ~
  
  \begin{exo}(4.5 points)
  
   Résoudre dans $\mathbb{R}$ :
   \begin{enumerate}
   
       \item $\frac{9}{2}x^2-5 x+\frac{2}{3}=0$
    
\begin{correction}
 
 Les solutions de cette \'equation sont les racines du trin\^ome $\frac{9}{2}x^2-5 x+\frac{2}{3}$
 dont le discriminant est $\Delta=b^2-4ac=(-5)^2-4\times \frac{9}{2} \times \frac{2}{3}=25-12=13$
 %$3/2(x^2-8/3 x+1)$, $\Delta=64/9-36/9=7$
    
    $x_1=\frac{-b-\sqrt{\Delta}}{2a}=\frac{5-\sqrt{13}}{2(\frac{9}{2})}=\frac{5-\sqrt{13}}{9}$
    
    $x_2=\frac{5+\sqrt{13}}{9}$
    
    d'o\`u $S=\left \{\frac{5-\sqrt{13}}{9};\frac{5+\sqrt{13}}{9}\right \}$ (1 pt)
 
\end{correction}
   
    %\item $(-2x+3)(x+2)=0$
    \item $(x+1)(x+3)=(4x+4)x$
    
\begin{correction}

$\begin{aligned}[t]
           & (x+1)(x+3)=(4x+4)x \\
      \iff \\
           & (x+1)(x+3)-4(x+1)x=0\\
      \iff \\
           & (x+1)(x+3-4x)=0\\
      \iff  \\
           &(x+1)(1-x)=0\\
      \iff  \\
          &x+1=0 \textrm{ ou } 1-x=0\\    
      \iff  \\
          &x=-1 \textrm{ ou } x=1\\      
\end{aligned}$
    
donc $S=\{-1;1\}$ (1 pt)
 
\end{correction}  
    
 
    \item $13 x^2-7 x+9>7$

\begin{correction}

$\begin{aligned}[t]
           & 13 x^2-7 x+9>7 \\
      \iff \\
           & 13 x^2-7 x+2>0\\
 \end{aligned}$          
 
 $13 x^2-7 x+2=0$ , $\Delta=b^2-4ac=49-4(13)(2)<0$, Il n'y a pas de racine.
 Ce trin\^ome est toujours strictement positif (a=13>0)
    
 et   $S=\mathbb{R}$ (1 pt)
    
 
\end{correction}   

  \item $\frac{5-3 x}{x-3}>x$.

\begin{correction}

$\begin{aligned}[t]
           & \frac{5-3x}{x-3}>x\\
      \iff \\
           & \frac{5-3x}{x-3}-x>0\\
      \iff \\
           & \frac{5-3x-x(x-3)}{x-3}>0\\
      \iff \\
           &\frac{-x^2+5}{x-3}>0 \textrm{ (0.5 pt pour la r\'eduction) }\\     
      \iff \\
           &\frac{-(x-\sqrt{5})(x+\sqrt{5})}{x-3}>0 \textrm{ (0.5 pt pour le calcul des racines) }\\    
\end{aligned}$
 
 Soit $f(x)=\frac{-(x-\sqrt{5})(x+\sqrt{5})}{x-3}$
%$\Delta=5$, $x_1=\frac{1-\sqrt{5}}{2}$, $x_2=\frac{1+\sqrt{5}}{2}$

\begin{tikzpicture}
\tkzTabInit{$x$/1,$x-3$/1,$x+\sqrt{5}$/1,$-(x-\sqrt{5})$/1,$f(x)$/1}{$-\infty$,$-\sqrt{5}$,$\sqrt{5}$,$3$,$+\infty$}
\tkzTabLine{,,,-,,,z,+}
\tkzTabLine{,-,z,,,+,,}
\tkzTabLine{,,+,,z,,-,}
\tkzTabLine{,+,z,-,z,+,d,-}
\end{tikzpicture}
 (1 pt) 
 
 d'o\`u $S=\left ]-\infty;-\sqrt{5}\right [ \cup \left ]\sqrt{5};3 \right [$ (0.5 pt)
\end{correction}   
   
    
    
    
    \end{enumerate}
  \end{exo}

  
  \vspace{0.5cm}
  ~
  
  \begin{exo}(4 points)
  
  \vspace{0.15cm}
    Soient $f_1(x)=(x+3)^2-2$ et $f_2(x)=-x^2+5x+7$ deux trinômes. 
    
    Soient $\mathcal{P}_1:y=f_1(x)$ et $\mathcal{P}_2:y=f_2(x)$
    leurs représentations graphiques.
    \begin{enumerate}
     \item Calculer les coordonnées du sommet des paraboles $\mathcal{P}_1$ et $\mathcal{P}_2$.

\begin{correction}
 
 $f_1(x)$ est sous forme canonique, on peut lire directement les coordonn\'ees du sommet $S(-3;-2)$ (1 pt)
 
 $\alpha=\frac{-b}{2a}=\frac{-5}{-2}=\frac{5}{2}$ et 
 $\beta=f_2(\frac{5}{2})=-(\frac{5}{2})^2+5(\frac{5}{2})+7=-\frac{25}{4}+\frac{50}{4}+\frac{28}{4}=\frac{53}{4}$ 
 et $S(\frac{5}{2};\frac{53}{4})$ (1 pt)
 
\end{correction}

   
    
    \item Dresser les tableaux de variations de $f_1$ et $f_2$.

\begin{correction}
 
  \[
      \begin{tikzpicture}
	\tkzTabInit{$x$ /1,$f_1(x)$/2}{$-\infty$, $-3$, $+\infty$}
	
	\tkzTabVar{+/$+\infty$,-/$-2$,+/$+\infty$}
      \end{tikzpicture}
      \textrm{ car }a=1>0\textrm{ (1 pt) }
      \] 
    
     \[
      \begin{tikzpicture}
	\tkzTabInit{$x$/1,$f_2(x)$/2}{$-\infty$, $\frac{5}{2}$, $+\infty$}
	
	\tkzTabVar{-/$-\infty$,+/$\frac{53}{4}$,-/$-\infty$}
      \end{tikzpicture}
      \textrm{ car }a=-1<0\textrm{ (1 pt) }
      \]
 
\end{correction}     
    
    
%    \item En déduire, sans justification, le nombre de racines de ces deux trinômes.

%\begin{correction}
 
%  Par lecture des tableaux de variations, $f_1$ n'admet pas de racine et $f_2$ admet
%    deux racines. (1 pt)
 
%\end{correction}
    
    
    
    
    \end{enumerate}
  \end{exo}
  
 
  \vspace{0.5cm}
  ~
  
  \begin{exo}(2.5 points)
  La parabole suivante est la représentation graphique d'un trinôme 
  
  $f(x)=ax^2+bx+c$ dont la forme canonique est $f(x)=a(x-\alpha)^2+\beta$. On note $\Delta$ le discriminant de $f(x)$.
  Donner sans justification le signe des paramètres 
  $a,c,\alpha,\beta,\Delta$ 
  pour le trinôme
  dont la représentation graphique est la suivante:
  
  \begin{center}
   \begin{tikzpicture}
    
    \draw[step=0.5cm,gray,very thin](-2.9,-2.4) grid (2.9,2.9);
    \draw[very thick,->](-2.5,-2)--(2.5,-2);
    \draw[very thick,->](-2,-2.5)--(-2,2.5);
    \draw[thick][domain=-2.9:1.1] plot (\x,{(\x+1)*(\x+1)-1.5});
    %(\x -0.5*(\x-3)(\x-3)+2);
    
   \end{tikzpicture}
   \end{center}

\begin{correction}
 
  $a>0$,$c>0$, $\alpha>0$, $\beta>0$ et $\Delta<0$. (0.5 pt par r\'eponse)
 
\end{correction}
  
  \end{exo}
    
  
  \vspace{0.5cm}
  ~
  
  \begin{exo}(3 points)
  
Déterminer toutes les valeurs du réél $m$ pour lesquelles l'équation $x^2+2mx+1=0$ admet une racine double.

\begin{correction}
 $x^2+2mx+1=0$ admet une racine double.
 
 \vspace{0.5cm}
 
$
\begin{aligned}[t]
           & x^2+2mx+2=0 \\ %\textrm{admet une racine double}\\
      \iff \\
           & \Delta=0\\
      \iff \\
           & 4m^2-8=0\\
      \iff \\
           & m^2-2=0\\
      \iff \\
           & m^2-(\sqrt{2})^2=0\\
      \iff \\
           & (m-\sqrt{2})(m+\sqrt{2})=0\\     
\end{aligned}
$

  $\sqrt{2}$ et $-\sqrt{2}$ sont les seules valeurs pour $m$ pour lesquelles le trin\^ome admet une racine double.
\end{correction}

    
  \end{exo}
  
  
%  \begin{exo}(2 points)
  
%  \vspace{0.15cm}
  
%   Trouver toutes les valeurs $t$ pour lesquelles le trinôme
%   $h(x)=b^2-4ac$
%  \end{exo}
  


\end{document}

  \vspace{0.5cm}
  ~
  
  \begin{exo}(3 points)
  
   Déterminer deux réels dont la somme est égale à 5 et le produit est égal à 2.

   
\begin{correction}
 
 
 Soient $x$ et $y$ deux r\'eels. 

 \iffalse
$\begin{aligned}[t]
           &
\begin{cases}
x+y=5\\
x y=2\\
\end{cases}\\
      \iff \\
  &
\begin{cases}
 
y=5-x\\
 
x(5-x)=2
 
\end{cases}\\
      
 \end{aligned}$
 
 
 
 
  
  

   $y=5-x$ et $x(5-x)=2$
   
   $-x^2+5x-2=0$, 

\fi
    
    \begin{tabular}{ >{$}l<{$} }
\left  \{ \begin{tabular}{ >{$}l<{$} } x+y=5  \\ xy=2 \end{tabular}\right.\\
\iff  \\
\left \{ \begin{tabular}{ >{$}l<{$} } y=5-x  \\ x(5-x)=2 \end{tabular}\right.\\
\end{tabular}
(0.5 pt pour la mise en \'equation) 
(1 pt pour la substitution)

or $x(5-x)=2 \iff -x^2+5 x-2=0$

$\Delta=25-4(-1)(-2)=17$, $x_1=\frac{-5-\sqrt{17}}{-2}=\frac{5+\sqrt{17}}{2}$
  $x_2=\frac{5-\sqrt{17}}{2}$ (1 pt pour le calcul des racines).
  
  On remarque $x_2=5-x_1$ et $x_1=5-x_2$. Il y a donc une unique paire de r\'eels
  dont la somme est \'egale \' a 5 et le produit est \'egal \`a 2. Il s'agit de la paire 
  $\left \{\frac{5+\sqrt{17}}{2},\frac{5-\sqrt{17}}{2}\right \}$. (0.5 pt pour la conclusion).

\end{correction}

   \end{exo}

 \vspace{0.5cm}
  ~
  
  \begin{exo}(2 points)
  
   Calculer les coordonnées des points d'intersection entre la droite $d$ d'équation
   $d:y=2x+2$ et la parabole $\mathcal{P}$ d'équation $\mathcal{P}:y=x^2-1$.
  \end{exo}
 
\begin{correction}
 
 \begin{tabular}{ >{$}l<{$} }
\left \{ \begin{tabular}{ >{$}l<{$} } y=2x+2  \\ y=x^2-1 \end{tabular}\right.\\
\iff  \\
\left \{ \begin{tabular}{ >{$}l<{$} }  y=2x+2 \\ 2x+2=x^2-1 \end{tabular}\right.\\
\iff  \\
\left \{ \begin{tabular}{ >{$}l<{$} }  y=2x+2 \\ x^2-2x-3=0 \end{tabular}\right.\\
\iff  \\
\left \{ \begin{tabular}{ >{$}l<{$} }  y=2x+2 \\ (x+1)(x-3)=0 \end{tabular}\right.\\
\iff  \\
\left \{ \begin{tabular}{ >{$}l<{$} }  y=2x+2 \\ x=-1 \textrm{ ou } x=3 \end{tabular}\right.\\
\iff  \\
  (x=-1\textrm{ et }y=0)\textrm{ ou }(x=3\textrm{ et }y=8)\\

\end{tabular}
(1 pt pour le calcul des racines)

Il y a deux points d'intersection entre $d$ et $\mathcal{P}$ qui sont les points
$A(-1;0)$ et $B(3;8)$. (1 pt pour la conclusion)
   
  
  
 
\end{correction}