\documentclass{beamer}

%\documentclass{article}
%\usepackage{beamerarticle}

\usepackage[utf8]{inputenc}

%\usetheme{Warsaw}
%\usetheme{Hannover}
\usetheme{Berkeley}
%\usecolortheme{lily}
\setbeamertemplate{theorems}[numbered] 
%\setbeamertemplate{theorems}[ams style]
\date{}

%\theoremstyle{plain}

\usepackage{lmodern}
\usepackage[T1]{fontenc}
\usepackage[utf8]{inputenc}
\usepackage[french]{babel}
\usepackage{tikz,tkz-tab}

\uselanguage{French}
\languagepath{French}

\newtheorem{proposition}[theorem]{\translate{Proposition}}
%\newtheorem{example}[theorem]{\translate{Example}}
\newtheorem{demonstration}[theorem]{Démonstration}

\title{Vecteurs et équations de droites}


\begin{document}
  
  \begin{frame}
    
    \titlepage
   % \maketitle
    
  \end{frame}
  
  
    
  \section{Vecteurs du plan}
  
  \subsection{Colinéarité}
  \begin{frame} 
  
  \begin{definition}    
    Deux vecteurs $\vec{u}$ et $\vec{v}$ sont \textbf{colinéaires} si et seulement si il existe 
    un réel \uncover<2,3,4>{$k$} tel que $\vec{u}=\uncover<3,4>{k\vec{v}}$ ou 
    \uncover<4>{$\vec{v}=k\vec{u}$}.
  \end{definition}
  
  \end{frame}
  
  
  \begin{frame}
    \begin{example}
    \begin{center}
   \begin{tikzpicture}
    
    \draw[step=0.5cm,gray,very thin](-5.9,0.4) grid (-1.6,3.9);
    \node [above] at (-4,2) {$\vec{u}$};
    \draw[very thick,->](-5.5,1.5)--(-2.5,2.5);
    \node [above] at (-2.75,1.25) {$\vec{v}$};
    \draw[very thick,->](-2,1.5)--(-3.5,1);
    \draw[very thick,->](0,1)--(0.5,3);
    \draw[thin,-](-0.25,0)--(0.625,3.5);
    \draw[thin,-](0.5,0)--(1.375,3.5);
    \draw[very thick,->](1,2)--(0.625,0.5);
    \node [left] at (0.1875,1.75) {$\vec{CD}$};
    \node [right] at (1,1) {$\vec{AB}$};
    \node [right] at (2,2) {$(AB) \slash{} \slash{} (CD)$};
    %\draw[very thick,->](-5.9,0.4)--(-0.5,3.9);
    %\draw[thick][domain=-2.9:1.1] plot (\x,{(\x+1)*(\x+1)-1.5});
    %(\x -0.5*(\x-3)(\x-3)+2);
    
   \end{tikzpicture}
      \begin{tabular}{p{5cm}p{4.7cm}}
$\vec{u}=\uncover<2,3>{-2} \vec{v}$
&
 $\vec{AB}=\uncover<3>{-\frac{AB}{CD}\vec{CD}}$
\end{tabular}
   \end{center}
   
  \end{example}
  \end{frame}
  
  \subsection{Décomposition de vecteur}
    
  \begin{frame}
    
  \begin{theorem}
    Tout vecteur du plan peut s'exprimer en fonction de deux vecteurs non colinéaires.
    
    Autrement dit,
    si $\vec{v}$ et $\vec{w}$ sont deux vecteurs non colinéaires alors pour tout vecteur $\vec{u}$,
    il existe un unique couple de réels $(a,b)$ tel que $\vec{u}=a\vec{v}+b\vec{w}$.
    
  \end{theorem}
  
    
  \end{frame}  
  
   \begin{frame}
    \begin{example}
    
    \begin{tikzpicture}
    
    \draw[step=0.5cm,gray,very thin](-5.9,0.4) grid (-1.6,3.9);
    \uncover<2,3,4,5>{\draw[semithick,-](-5.8,0.4)--(-1.6,1.8);}
    \uncover<3,4,5>{\draw[dashed,semithick,-](-5.8,2.4)--(-1.6,3.8);}
    \uncover<2,3,4,5>{\draw[thick,-](-2.5,0.4)--(-2.5,3.9);}
    \uncover<3,4,5>{\draw[dashed,thick,-](-5.5,0.4)--(-5.5,3.9);}
    \node [right] at (-2.5,1) {$\vec{v}$};
    \draw[ultra thick,->](-2.5,1.5)--(-2.5,0.5);
    \uncover<4,5>{\draw[ultra thick,->](-2.5,1.5)--(-2.5,3.5);}
    \uncover<4,5>{\node [right] at (-2.5,2.5) {$-2 \vec{v}$};}
    \node [below] at (-3.275,1.25) {$\vec{w}$};
    \draw[very thick,->](-2.5,1.5)--(-4,1);
    \uncover<5>{\draw[very thick,->](-2.5,1.5)--(-5.5,0.5);}
    \node [above] at (-4,2) {$\vec{u}$};
    \draw[very thick,->](-2.5,1.5)--(-5.5,2.5);
    \draw[very thick,->](2,1)--(4,1);
    \draw[very thick,->](2,1)--(3,3);
    \draw[thin,-](0,1)--(4.3,1);
    %\draw[thin,-](0.5,0)--(1.375,3.5);
    \node [below] at (2,1) {$A$};
    \node [below] at (4,1) {$B$};
    \node [below] at (1,1) {$C$};
    \node [below] at (3,1) {$D$};
    \node [above] at (3,3) {$E$};
    \draw[very thick,-](2,0.9)--(2,1.1);
    \draw[very thick,->](2,1)--(1,1);
    \draw[very thick,-](3,0.9)--(3,1.1);
    
    
    
    
    
   \end{tikzpicture}
   
    \begin{tabular}{p{4.7cm}|p{4.7cm}}
$\vec{u}=\uncover<4,5>{-2 }\vec{v}+\uncover<5>{2} \vec{w}$
&
 $\vec{AD}$
 
 $=0 \vec{AB}+(-1)\vec{AC}$
 
 $=1 \vec{AB}+1\vec{AC}$
\end{tabular}

    
  \end{example}
  \end{frame}
  
  \subsection{Repères du plan}
  
  \begin{frame}
   \begin{definition}
      Un \textbf{repère} du plan est la donnée d'un point $O$, appelé origine du repère, et de deux vecteurs
      $\vec{i}$ et $\vec{j}$ non colinéaires. Il se note $(O,\vec{i},\vec{j})$.
   \end{definition}
   
   \end{frame}
  
  \begin{frame}
   \begin{example}
    
     \begin{tikzpicture}
    
    \draw[step=0.5cm,gray,very thin](-5.9,0.4) grid (-1.6,3.9);
    \node [below] at (-4,1.5) {$O$};
    \node [below] at (-3.5,1.5) {$\vec{i}$};
    \node [left] at (-4,2) {$\vec{j}$};
    \draw[very thick,->](-4,1.5)--(-3,1.5);
    \draw[very thick,->](-4,1.5)--(-4,2.5);
    \node [right] at (3,1.5) {$A$};
    \draw[very thick,->](3,1.5)--(0.5,1);
    \node [left] at (0.5,1) {$B$};
    \draw[very thick,->](3,1.5)--(0,2.5);
    \node [left] at (0,2.5) {$C$};
 
    
   \end{tikzpicture}
   
   \begin{tabular}{p{4.7cm}|p{4.7cm}}
$(O,\vec{i},\vec{j})$ est un repère 

\uncover<2>{orthonormé}.
&
 $(A;\vec{AB},\vec{AC})$ est un repère quelconque.
\end{tabular}
   
   
   \end{example}
  
  \end{frame}

  \subsection{Systèmes de coordonnées}
  
  \begin{frame}
   \begin{proposition}
    \'Equivalence fondamentale: 
    
    Un point $M$ du plan a pour coordonnées $(x;y)$ dans le repère $(O;\vec{i},\vec{j})$
    
    si et seulement si $\vec{OM}$ a pour coordonnées $(x;y)$ 

    si et seulement si $\vec{OM}=x \vec{i}+y \vec{j}$.
  
   \end{proposition}


  \end{frame}
  
  \begin{frame}
   \begin{example}
    
    \begin{tikzpicture}
    
    \draw[step=0.5cm,gray,very thin](-5.9,0.4) grid (-1.6,3.9);
    \draw[very thick,-](-3,2.9)--(-3,3.1);
    \draw[very thick,-](-2.9,3)--(-3.1,3);
    \node [above] at (-3,3) {$M$};
    \node [below] at (-4,1.5) {$O$};
    \node [below] at (-3.5,1.5) {$\vec{i}$};
    \node [left] at (-4,2) {$\vec{j}$};
    \draw[very thick,->](-4,1.5)--(-3,1.5);
    \draw[very thick,->](-4,1.5)--(-4,2.5);
    
    \node [right] at (3,1.5) {$A$};
    \draw[very thick,-](3,1.5)--(0.5,1);
    \node [left] at (0.5,1) {$B$};
    \draw[very thick,-](3,1.5)--(0,2.5);
    \node [left] at (0,2.5) {$C$};
    \draw[very thick,-](0.5,1)--(0,2.5);
    \draw[very thick,-](0.15,1.75)--(0.35,1.75);
    \node [left] at (0.25,1.75) {$I$};
    \draw[thin,-](0.25,1.75)--(1.5,2);
    \draw[thin,-](0.25,1.75)--(1.75,1.25);
    \node [above] at (1,3) {$I$ est le milieu de $[BC]$.};
    
   \end{tikzpicture}
   
     \begin{tabular}{p{4.7cm}|p{4.7cm}}
  $M$ a pour coordonnées $M(\uncover<2,3>{1};\uncover<2,3>{1,5})$ dans le repère $(0;\vec{i},\vec{j})$.
&
  $\vec{AI}= \frac{1}{2} \vec{AB}+\frac{1}{2} \vec{AC}$ et 
   $I(\uncover<3>{\frac{1}{2}};\uncover<3>{\frac{1}{2}})$ dans le repère
   $(A;\vec{AB},\vec{AC})$.
\end{tabular}

    
   \end{example}

  \end{frame}
  
  \subsection{Critère de colinéarité}
  
  \begin{frame}
  
  \begin{proposition}
   Soient $\vec{u}$ et $\vec{u'}$ deux vecteurs de coordonnées $\vec{u}(x;y)$ et $\vec{u'}=(x';y')$
   dans un repère.
   
   \begin{center}
      $\vec{u}$ et $\vec{u'}$ sont colinéaires si et seulement si $\uncover<2>{xy'-x'y}=0$.
   \end{center}

   
  \end{proposition}
  \end{frame}
  
  \begin{frame}
   
   \begin{example}

    Soit $(O;\vec{i},\vec{j})$ un repère. Les vecteurs suivants sont-ils colinéaires ? 
   
   \begin{tabular}{p{4.7cm}|p{4.7cm}}
   
   
$\vec{u}(1;2)$ et $\vec{v}(2;4)$

\uncover<2,3,4,5>{$1 \times 4- 2 \times 2 = 0$}

\uncover<3,4,5>{$\vec{u}$ et $\vec{v}$ sont colinéaires}.

& 

$\vec{w}(2;3)$ et $\vec{z}(5;7)$

\uncover<4,5>{$2 \times 7- 3 \times 5 = 14-15=-1$}

\uncover<5>{$\vec{w}$ et $\vec{z}$ ne sont pas colinéaires}.\\
 \\

\end{tabular}

\end{example}

\end{frame}


  \section{Droites et vecteurs directeurs}
  
  \subsection{Vecteurs directeurs d'une droite}
  
  
  


    \begin{frame}
   \begin{definition}
    
   On dit qu'un vecteur $\vec{v}$ est un \textbf{vecteur directeur} d'une droite $\mathcal{D}$ si il existe
   deux points $A$ et $B$ de $\mathcal{D}$ tels que $\vec{v}=\uncover<2>{\vec{AB}}$.
  \end{definition}
  \end{frame}
  
  \begin{frame}
  
   \begin{example}
    
      \begin{tikzpicture}
    
    \draw[step=0.5cm,gray,very thin](-5.9,0.4) grid (-1.6,3.9);
    
    \node [above] at (-3,3) {$B$};
    \node [below] at (-4,1.5) {$A$};
    
    
    \uncover<2,3,4,5,6,7,8>{\draw[very thick,|->](-4,1.5)--(-3,3);}
    \uncover<1>{\draw[thin,|-|](-3,3)--(-4,1.5);}
    \draw[thin,-](-2.4,3.9)--(-4.73,0.4);
    
    \node [below] at (3,1.5) {$C$};
    \node [above] at (2.7,1.8) {$\vec{u}$};
    \uncover<5,6,7,8>{\node [above] at (1,2.17) {$\mathcal{D}$};}
    
    \draw[very thick,|->](3,1.5)--(2,1.83);
    \uncover<5,6,7,8>{\draw[thin,-](-1,2.83)--(4,1.17);}
    
   \end{tikzpicture}
    
     \begin{tabular}{p{4.7cm}|p{4.7cm}}
   
\uncover<3,4,5,6,7,8>{Le vecteur }\uncover<4,5,6,7,8>{$\vec{AB}$} 
\uncover<3,4,5,6,7,8>{ est un vecteur directeur de la droite $(AB)$.}

& 

\uncover<6,7,8>{$\mathcal{D}$ est la droite passant par le point} \uncover<7,8>{$C$} 
\uncover<6,7,8>{et dirigée
par le vecteur} \uncover<8>{$\vec{u}$}\uncover<6,7,8>{.}


\end{tabular}
    
   \end{example}
   
   \end{frame}
   
   \begin{frame}
     \begin{proposition}
   Soit $\mathcal{D}$ une droite de vecteur directeur $\vec{u}$.
   
   Les vecteurs directeurs de $\mathcal{D}$ sont tous les vecteurs non nuls colinéaires à \uncover<2>{$\vec{u}$}.
  \end{proposition}
   
   \end{frame}
   
     \begin{frame}
  
   \begin{example}
    
      \begin{tikzpicture}
    
    \draw[step=0.5cm,gray,very thin](-5.9,0.4) grid (-1.6,3.9);
    
    \node [above] at (-3,3.1) {$B$};
    \node [above] at (-4.4,1.6) {$A$};
    
    
    \draw[very thick,|->](-4,1.5)--(-3,3);
    \uncover<2>{\draw[very thick,|->](-4.1,1.35)--(-4.5,0.75);}
    \uncover<3>{\draw[very thick,|->](-3.5,2.25)--(-3.25,2.63);}
    \uncover<4>{\draw[very thick,|->](-2.8,3.3)--(-2.53,3.7);}
    \uncover<5>{\draw[very thick,|->](-3.8,3.3)--(-3.53,3.7);}


    %\node [above] at (-3.25,2.73) {$C$};
    %\draw[very thick,->](-4,1.5)--(-2.3,4.05);
    %\draw[very thick,->](-4,1.5)--(-3.5,2.25);
    %\draw[very thick,->](-4,1.5)--(-4.25,1.13);
    
    \draw[thin,-](-2.4,3.9)--(-4.73,0.4);
    \draw[thin,-](-2.4,3.9)--(-4.73,0.4);
      
    \node [below] at (3,1.5) {$C$};
    \node [above] at (2.7,1.8) {$\vec{u}$};
    \node [above] at (1,2.17) {$\mathcal{D}$};
    
    \draw[very thick,|->](3,1.5)--(2,1.83);
    \uncover<6>{\draw[very thick,|->](1,2.16)--(-1,2.82);}
    \uncover<7>{\draw[very thick,|->](3.3,1.4)--(3.6,1.3);}
    \draw[thin,-](-1,2.83)--(4,1.17);
    
   \end{tikzpicture}
   
   \end{example}
   
   \end{frame}
   
   \subsection{Parallélisme et vecteurs directeurs}
   
   \begin{frame}
   \begin{theorem}
    Deux droites sont parallèles si et seulement si elles ont
    des vecteurs directeurs \uncover<2,3>{colinéaires}.
    
    Aurement dit, $\mathcal{D}$ et $\mathcal{D}'$ de vecteurs directeurs respectifs $\vec{u}$
    et $\vec{u}'$ sont parallèles si et seulement si \uncover<3>{$\vec{u}$} et \uncover<3>{$\vec{u}'$} sont colinéaires.
   \end{theorem}


  \end{frame}

       \begin{frame}
  
   \begin{example}
    
      \begin{tikzpicture}
    
    \draw[step=0.5cm,gray,very thin](-5.9,0.4) grid (-1.6,3.9);
    
    \node [below] at (-3,2.9) {$B$};
    \node [below] at (-4,1.5) {$A$};
    
    \uncover<3,4,5,6,7,8>{\node [above] at (-3.25,3.37) {$C$};}
    \uncover<3,4,5,6,7,8>{\node [above] at (-3.75,2.72) {$D$};}
    
    
    \draw[very thick,|->](-4,1.5)--(-3,3);
    \uncover<3,4,5,6,7,8>{\draw[very thick,|->](-3.25,3.37)--(-3.75,2.62);}
    \draw[thin,-](-2.4,3.9)--(-4.73,0.4);
    \uncover<2,3,4,5,6,7,8>{\draw[thin,-](-2.9,3.9)--(-5.23,0.4);}
    
    \node [below] at (3,1.5) {$C$};
    \node [above] at (2.7,1.8) {$\vec{u}$};
    \uncover<6,7,8>{\node [above] at (2.7,2.8) {$\mathcal{D}'$};}
    \node [above] at (1,2.17) {$\mathcal{D}$};
    \uncover<6,7,8>{\node [above] at (1,3.15) {$\vec{u'}$};}
    
    \draw[very thick,|->](3,1.5)--(2,1.83);
    \draw[thin,-](-1,2.83)--(4,1.17);
    \uncover<6,7,8>{\draw[very thick,|->](0,3.49)--(2,2.83);}
    \uncover<6,7,8>{\draw[thin,-](-1,3.83)--(4,2.17);}
    
   \end{tikzpicture}
   
        \begin{tabular}{p{4.7cm}|p{4.7cm}}
   
\uncover<4,5,6,7,8>{La droite $(AB)$ est parallèle à la droite $(CD)$

si et seulement si ($\iff$) 

le vecteur $\vec{AB}$ est} \uncover<5,6,7,8>{colinéaire}
\uncover<4,5,6,7,8>{au vecteur $\vec{CD}$.}

& 

\uncover<7,8>{Les droites $\mathcal{D}$ et $\mathcal{D}'$ sont parallèles si et seulement si
les vecteur $\vec{u}$ et $\vec{u}$ sont }\uncover<8>{colinéaires}\uncover<7,8>{.}

\end{tabular}
   
   \end{example}
   \end{frame}
   
   %\iffalse
   
      \subsection{Appartenance d'un point à une droite}
 
  
 
   \begin{frame}
   
   
   \begin{proposition}
    Un point $M$ appartient à la droite passant par $A$ et de vecteur directeur $\vec{u}$
    si et seulement si le vecteur \uncover<2>{$\vec{AM}$} est colinéaire au vecteur $\vec{u}$.
   \end{proposition}


  \end{frame}

  \begin{frame}
  
  \begin{example}
  

       \begin{tikzpicture}
    
    \draw[step=0.5cm,gray,very thin](-5.9,0.4) grid (-1.6,3.9);
    
    \node [above] at (-3,3) {$B$};
    \node [right] at (-4,1.5) {$A$};
    \uncover<2,3,4,5,6>{\node [left] at (-4.5,0.75) {$M$};}
    
    
    \draw[very thick,|->](-4,1.5)--(-3,3);
    \uncover<2,3,4,5,6>{\draw[very thick,->](-4,1.5)--(-4.5,0.75);}
    \draw[thin,-](-2.4,3.9)--(-4.73,0.4);
    
    \node [below] at (3,1.5) {$C$};
    \node [above] at (2.7,1.8) {$\vec{u}$};
    \uncover<4,5,6>{\node [above] at (1,2.17) {$M$};}
    
    \draw[very thick,|->](3,1.5)--(2,1.83);
    \uncover<4,5,6>{\draw[very thick,->](3,1.5)--(1,2.16);}
    \draw[thin,-](-1,2.83)--(4,1.17);
    
   \end{tikzpicture}
   
\begin{tabular}{p{4.7cm}|p{4.7cm}}
Les points $A$, $B$ et $M$ sont alignés si et seulement si
\uncover<3,4,5,6>{$\vec{AM}$} est colinéaire au vecteur $\vec{AB}$.
&
Le point $M$ appartient à la droite passant par le point $C$ et dirigée par $\vec{u}$
si et seulement si le vecteur \uncover<5,6>{$\vec{CM}$} est colinéaire au vecteur \uncover<6>{$\vec{u}$}.
\end{tabular}
  \end{example}
\end{frame}
   
   \section{\'Equations de droites}
   
   
   
   \subsection{\'Equation cartésienne de droite}
   
    \begin{frame}
    
     Jusqu'à la fin de ce cours, le plan est rapporté à un repère $(O;\vec{i},\vec{j})$.
   
     \begin{theorem}
     
    

    Soient $a,b$ deux réels tels que l'un au moins des nombres $a$ et $b$ est non nul.
    
    Toute droite du plan de vecteur directeur $\vec{u}(-b;a)$ admet une équation de la forme 
    $$ax+by+c=0, \textrm{avec $c \in \mathbb{R}$}$$
   
   \end{theorem}

  \end{frame}
  
  \begin{frame}
  \begin{demonstration}
    
    Soit $\mathcal{D}$ une droite de vecteur directeur $\vec{u}(-b;a)$ et $A(x_A;y_A)$ un point de 
    $\mathcal{D}$. Soit $M(x;y)$ un point du plan.
    
    $M \in \mathcal{D}$
    
    $\iff \uncover<2,3,4,5,6>{\vec{AM}}(\uncover<3,4,5,6>{x-x_A};\uncover<3,4,5,6>{y-y_A}) 
    \textrm{ est colinéaire à } \uncover<2,3,4,5,6>{\vec{u}(-b;a)}$
    
    $\iff (x-x_A)\times \uncover<4,5,6>{a} -\uncover<5,6>{(y-y_A)} \times (-b)=0$ 
    
    $\iff ax+by+c=0 \textrm{ en posant }c=\uncover<6>{-ax_A-by_A}$.
    
    La droite $\mathcal{D}$ admet donc bien une équation de la forme $ax+by+c=0$.
   \end{demonstration}
   
   \end{frame}
   
   \begin{frame}
    
      
   \begin{theorem}

    Soient $a,b,c$ trois réels tels que l'un au moins des nombres $a$ et $b$ est non nul.
    
    L'ensemble des points $M(x;y)$ dont les coordonnées vérifient l'équations
    $$ax+by+c=0$$ est une droite $\mathcal{D}$ de vecteur directeur $\vec{u}\uncover<2>{(-b;a)}$.
    
    L'équation $ax+by+c=0$ est appelée une \textbf{équation cartésienne} de la droite $\mathcal{D}$.
   
   \end{theorem}

  \end{frame}
   
   \begin{frame}
     \begin{definition}
   
   Une équation de droite de la forme $\mathcal{D}:\uncover<2>{ax+by+c=0}$ est appelée une \textbf{équation cartésienne} de la droite $\mathcal{D}$.
   \end{definition}
   \end{frame}

  
       \begin{frame}
    
   \begin{example}
    Soit $\mathcal{D}:2 x+3 y-4=0$
      \begin{tikzpicture}[scale=0.5]
    
    \draw[step=1cm,gray,very thin](-10,-3) grid (10,5);
    
    \draw[very thick,->](0,0)--(1,0);
    \draw[very thick,->](0,0)--(0,1);
    \draw[thick,-](-5.5,5)--(6.5,-3);
    \node [left] at (-4,3.8) {$A$};
    \node [left] at (-1,1.8) {$B$};
    \node [left] at (0,-0.4) {$O$};
    \node [below] at (0.5,0) {$\vec{i}$};
    \node [left] at (0,0.5) {$\vec{j}$};
    
    \draw[very thick,|-|](-1,2)--(-4,4);
    \draw[very thick,|->](5,-2)--(2,0);
    \draw[very thick,-](5,-2)--(2,-2);
    \draw[very thick,-](2,-2)--(2,0);
    
   \end{tikzpicture}
   
   $6x+9y-12=0$ est aussi une équation pour $\mathcal{D}$.
   
   $A(-4;4) \in \mathcal{D} \iff \uncover<2,3,4,5,6>{2(-4)+3(4)-4=0} \iff \uncover<3,4,5,6>{0=0}$
   
   
   $B(-1;2) \in \mathcal{D} \iff \uncover<4,5,6>{2(-1)+3(2)-4=0} \iff \uncover<5,6>{0=0}$
   
  
   $\vec{BA}(-4-(-1),4-2)=(-3;2)$ est un \uncover<6>{vecteur directeur} de $\mathcal{D}$.
   \end{example}
   
   \end{frame}
  
  \subsection{\'Equation réduite de droite}

  \begin{frame}
   
     \begin{theorem}
     
     \begin{itemize}
      \item 
      $\mathcal{D}$ est une droite du plan non parallèle à l'axe des ordonnées si et seulement si
    $\mathcal{D}$ admet une \uncover<2,3,4,5>{unique} équation réduite de la forme $y=mx+p$, où $m$ et $p$ sont des réels.
    
    Un vecteur directeur de $\mathcal{D}$ est $\vec{u}\uncover<3,4,5>{(1;m)}$, où $m$ est le coefficient directeur
    de la droite.
    
    \item
    $\mathcal{D}$ est une droite du plan parallèle à l'axe des ordonnées si et seulement si
    $\mathcal{D}$ admet une unique équation réduite de la forme \uncover<4,5>{$x=k$}, où $k$ est un réel.
    
    Un vecteur directeur de $\mathcal{D}$ est $\vec{j}(0;1)$.
    
    \item Deux droites non parallèles à l'axe des ordonnées sont parallèles si et seulement si 
    elles ont même \uncover<5>{coefficient directeur}.
     \end{itemize}
     
   \end{theorem}

  \end{frame}
  
\end{document}

$\begin{pmatrix}
     x\\
     y
    \end{pmatrix}$